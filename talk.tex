\documentclass[aspectratio=169]{beamer}
\usetheme[logo]{hypro} % the template, I can adjust it to WWU, if you want
\usepackage[useregional]{datetime2} % nice date typesetting
\usepackage{csquotes} % to quote stuff via \enquote{stuff}
\usepackage{xspace} % automated spacing after macros
\usepackage[normalem]{ulem} % allows to strikethrough text via \sout{text}

\title{Optimizing reachability probabilities for a restricted class of Stochastic Hybrid Automata via Flowpipe-Construction}
\author[Stefan Schupp]{Carina Pilch, \textbf{Stefan Schupp}, Anne Remke}
\date{\DTMdisplaydate{2021}{7}{7}{-1}}

%################ Commands
\newcommand{\cb}[1]{\textcolor{blue100}{#1}\xspace}



\begin{document}

\begin{frame}[plain]
\maketitle
\end{frame}

%########################################################

\begin{frame}{The problem in a nutshell}
  \enquote{\alert<5>{Optimizing \alert<4>{reachability probabilities}} for a \alert<3>{restricted class of \alert<2>{Stochastic \alert<1>{Hybrid Automata}}} via \alert<6>{Flowpipe-Construction}}

  \bigskip

  \only<1>{Hybrid automata: a model for \emph{hybrid systems}

  \cb{\faArrowRight} mixed discrete-continuous behavior
  }%
  \only<2>{Stochastic hybrid automata: a model for \emph{stochastic} hybrid systems

  \cb{\faArrowRight} add stochasticity
  \begin{itemize}
    \item continuous behavior: stochastic differential equations
    \item discrete behavior: stochastic jumps
  \end{itemize}
  }%
  \only<3>{Stochastic hybrid automata: a model for \emph{stochastic} hybrid systems

  \cb{\faArrowRight} add stochasticity
  \begin{itemize}
    \item \sout{continuous behavior: stochastic differential equations}
    \item discrete behavior: stochastic \alert<3>{urgent} jumps
    \item \cb{\emph{linear hybrid automata}}
    \begin{itemize}
      \item constant derivatives
      \item predicates compare variables to constants (\enquote{axis-aligned})
    \end{itemize}
  \end{itemize}
  \medskip

  \cb{Origin:} stochastic hybrid Petri nets with general transitions
  }%
  \only<4>{\cb{Question:} What is the probability, given a set of initial states of reaching a set of \emph{goal states}?}%
  \only<5>{\cb{Question:} What is the \alert{maximal/minimal} probability, given a set of initial states of reaching a set of \emph{goal states}?}%
  \only<6>{The method.}%
\end{frame}

%########################################################

\begin{frame}{The approach}
  \cb{Idea:} combination of techniques for purely hybrid systems and techniques for the analysis of stochastic systems

  \begin{enumerate}
    \item Compute forwards reachability
    \begin{itemize}
      \item Treat stochastic variables as clocks
      \item Keep track of \enquote{what you did} $\rightarrow$ \emph{reach tree}
    \end{itemize}
    \item Collect paths which lead to a goal state
    \item Compute \emph{optimal} probability of those paths with respect to different \emph{schedulers}
  \end{enumerate}

  \begin{reminder}{Scheduler}{5cm}
    Intuition: a scheduler resolves non-determinism

    Different types, relevant here:
    \begin{itemize}
      \item History-dependent (has memory),
      \item Prophetic (can predict the future) vs.
      \item Non-prophetic (decisions based on current knowledge)
    \end{itemize}
  \end{reminder}
\end{frame}

%########################################################

\begin{frame}{Flowpipe construction \& reach tree}

\end{frame}

%########################################################

\begin{frame}{Projection on random clocks}

\end{frame}

%########################################################

\begin{frame}{Optimal paths}

\end{frame}

%########################################################

\begin{frame}{Numerical integration: Monte Carlo simulation}

\end{frame}

%########################################################

\begin{frame}{Conclusion \& prospect}

\end{frame}

%########################################################

\end{document}
